\documentclass{book}

\usepackage{etoolbox}
\usepackage{xkeyval}
\usepackage{pgfkeys}
\usepackage{options}

\begin{document}
\makeatletter

\newcommand\preparetest[1]{
	\eifstrequal{#1}{options}{%
		\optionset{%
			/baz/bar-test/.new value = hi,
			/foo/.new family={/baz},
		}
		\def\test##1{\optionset{/foo,##1}}
		\def\valueof##1##2{\option{/##1/##2}}
	}
	{\eifstrequal{#1}{pgfkeys}{%
		\pgfkeys{%
			/baz/bar-test/.initial=hi,
			/foo/.is family,
			/foo/.search also={/baz},
		}
		\def\test##1{\pgfkeys{/foo,##1}}
		\def\valueof##1##2{\pgfkeysvalueof{/##1/##2}}
	}
	{\eifstrequal{#1}{xkeyval}{%
		\define@cmdkey{baz}{bar-test}{}%
		\define@cmdkey{foo}{x}{}%
		\def\test##1{\setkeys{foo,baz}{##1}}%
		\def\valueof##1##2{\csuse{cmdKV@##1@##2}}%
	}
	{\@latex@error{Cannot test package "#1"}}%
	}}%
}


\newcount\testcount
\newcommand\testten{%
  \test{bar-test={a test}}%
  \test{bar-test={a test}}%
  \test{bar-test={a test}}%
  \test{bar-test={a test}}%
  \test{bar-test={a test}}%
  \test{bar-test={a test}}%
  \test{bar-test={a test}}%
  \test{bar-test={a test}}%
  \test{bar-test={a test}}%
  \test{bar-test={a test}}%
  \test{bar-test={a test}}%  
  \advance\testcount 1\relax
}

\newcommand\runtest[2][10000]{%
	\preparetest{#2}%
	\typeout{test "#2": (\the\numexpr#1 * 10\relax\space repetitions)}
	\testcount=0%	
	\@whilenum\testcount<#1\do\testten
	\typeout{test "#2": "\valueof{baz}{bar-test}" (expected "a test")}
}

\iffalse
%\runtest{options}
%\runtest{pgfkeys}
%\runtest{xkeyval}

\else
% output benchmark results 
\def\plaintime{820}
\newcommand\adjust[2][1]{\the\numexpr#2 - \plaintime\relax ms (\the\numexpr (#2 - \plaintime)/#1\relax ms per 1000)}

\begin{tabular}{lrr}
\textbf{package} & \textbf{50.000x} & \textbf{100.000x} \\ 
\hline
options & \adjust[50]{3300} & \adjust[100]{5960} \\
pgfkeys & \adjust[50]{5520} & \adjust[100]{10840} \\
xkeyval & \adjust[50]{14500} & \adjust[100]{27900} \\
\hline
\multicolumn{3}{l}{baseline = \plaintime ms} \\
\multicolumn{3}{l}{platform = HP XW4600, Intel Core2 Quad CPU @ 3Ghz, 4Gb} \\
\multicolumn{3}{l}{latex    = XeTeX 3.1415926-2.5-0.9999.3-2013060713 (TeX Live 2013/W32TeX)}\\
\end{tabular}
\fi

\end{document}
