\documentclass{book}

\usepackage{etoolbox}
\usepackage{xkeyval}
\usepackage{pgfkeys}
\usepackage{options}

\begin{document}
\makeatletter



\options{%
  /bench/.new family,
  /bench/style/.new choice    = {simple,complex},
  /bench/library/.new choice  = {baseline,options,pgfkeys,xkeyval},
  /bench/repetitions/.new num = 10000,
  /bench/baseline/.new num    = 800,
  /bench/trace/.new toggle,
}

\options{/bench, library=baseline, style=complex, repetitions=50000}


\newcommand\preparetest[1]{
	\eifstrequal{#1}{options}{%
		\options{%
			/bar/bar-test/.new value = hi,
      /foo/foo-test/.new value = world,
			/foo/.new family={/bar},
		}    
		\def\test##1{\options{/foo,##1}}
		\def\valueof##1##2{\option{/##1/##2}}
	}
	{\eifstrequal{#1}{pgfkeys}{%
		\pgfkeys{%
			/bar/bar-test/.initial=hi,
      /foo/foo-test/.initial=world,
			/foo/.is family,
			/foo/.search also={/bar},
		}
		\def\test##1{\pgfkeys{/foo,##1}}
		\def\valueof##1##2{\pgfkeysvalueof{/##1/##2}}
	}
	{\eifstrequal{#1}{xkeyval}{%
		\define@cmdkey{bar}{bar-test}{}%
    \define@cmdkey{foo}{foo-test}{}%
		\define@cmdkey{foo}{x}{}%
		\def\test##1{\setkeys{foo,bar}{##1}}%
		\def\valueof##1##2{\csuse{cmdKV@##1@##2}}%
	}
	{\@latex@error{Cannot test package "#1"}}%
	}}%
}



\newcount\testcount
\newcommand\testtencomplex{%
  \test{bar-test={a test}}%
  \test{bar-test={a test}}%
  \test{bar-test={a test}}%
  \test{bar-test={a test}}%
  \test{bar-test={a test}}%
  \test{bar-test={a test}}%
  \test{bar-test={a test}}%
  \test{bar-test={a test}}%
  \test{bar-test={a test}}%
  \test{bar-test={a test}}%
  \test{bar-test={a test}}%  
  \advance\testcount -10\relax
}

\newcommand\testtensimple{%
  \test{foo-test={a test}}%
  \test{foo-test={a test}}%
  \test{foo-test={a test}}%
  \test{foo-test={a test}}%
  \test{foo-test={a test}}%
  \test{foo-test={a test}}%
  \test{foo-test={a test}}%
  \test{foo-test={a test}}%
  \test{foo-test={a test}}%
  \test{foo-test={a test}}%
  \test{foo-test={a test}}%
  \advance\testcount -10\relax
}

\newcommand\runtest[2][\option{/bench/repetitions}]{%
	\preparetest{#2}%
  \iftoggle{/bench/trace}
   {{\tracingmacros 1\relax\test{bar-test={a test}}}}
   {\typeout{test \option{/bench/style} "#2": (#1\space repetitions)}
  	\expandafter\let\expandafter\testproc\csname testten\option{/bench/style}\endcsname 
    \testcount=#1\relax%	    
    \@whilenum\testcount>0\do\testproc
  	\typeout{test "#2": " \valueof{bar}{bar-test}, \valueof{foo}{foo-test}"}%
   }
}

\ifcase\option{/bench/library/@ord}\relax
  % baseline = nothing
  \typeout{baseline: do nothing}%
\or
  \runtest{options}
\or
  \runtest{pgfkeys}
\or
  \runtest{xkeyval}
\else
  \optionerror[/bench/library]{unknown library}
\fi

\iffalse
% output benchmark results 
\newcommand\adjust[2][1]{\the\numexpr#2 - \option{/bench/baseline}\relax ms (\the\numexpr (#2 - \option{/bench/baseline}/#1\relax ms per 1000)}

\begin{tabular}{lrr}
\textbf{package} & \textbf{50.000x} & \textbf{100.000x} \\ 
\hline
options & \adjust[50]{3300} & \adjust[100]{5960} \\
pgfkeys & \adjust[50]{5520} & \adjust[100]{10840} \\
xkeyval & \adjust[50]{14500} & \adjust[100]{27900} \\
\hline
\multicolumn{3}{l}{baseline = \plaintime ms} \\
\multicolumn{3}{l}{platform = HP XW4600, Intel Core2 Quad CPU @ 3Ghz, 4Gb} \\
\multicolumn{3}{l}{latex    = XeTeX 3.1415926-2.5-0.9999.3-2013060713 (TeX Live 2013/W32TeX)}\\
\end{tabular}
\fi

\end{document}
